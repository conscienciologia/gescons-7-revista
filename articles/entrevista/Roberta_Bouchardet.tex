\documentclass{gescons}

\genre {Entrevista}
\author{Roberta Bouchardet}
\title{Interassistência: Teoria e Prática Sob a Ótica da Conscienciologia}

\begin{document}
    \makeentrevistatitle
    \coverart{back/Roberta_Bouchardet}

    \begin{multicols}{2}

\begin{center}
    \includegraphics[width=7cm]{articles/entrevista/mockups/Roberta-Bouchardet.png}
\end{center}

%\begin{center}
%    \includegraphics[width=8cm]{articles/entrevista/mockups/Roberta-Bouchardet.png}
%\end{center}

\textbf{1.       Qual foi a motivação para a escrita da obra? Por que a definição deste tema para publicação de um livro?}

Sempre gostei de livros e de escrever. E assim, junto com o incentivo do Prof. Waldo em deixar uma gestação consciencial nessa existência, veio a motivação de registrar parte do conhecimento adquirido em tantas décadas de estudo e autopesquisas na Conscienciologia. A definição do tema veio aos poucos. Começou com o verbete Autoamparo, que deveria ser a base do livro, mas a temática cresceu e se tornou Interassistência, sendo o Autoamparo um dos capítulos do livro.

\textbf{2.       Quais foram as principais percepções, intra e extrafísicas, durante a pesquisa e a escrita da obra? E posterior ao lançamento?}

Já há alguns anos eu estava escrevendo alguns capítulos, sem saber ainda o escopo da obra, e no Acoplamentarium de pesquisa, durante uma prática energética, tive o fenômeno de captação de ideias e visualizei todo o sumário, os capítulos a serem escritos. Anotei tudo e a partir daí a escrita fluiu bem mais rapidamente. Durante a escrita, sentia-me muito motivada e energizada, o mentalsoma não cansava. O que cansava era o soma, que exigia parar. Após o lançamento, percebo mudanças pensênicas, maior tranquilidade, talvez novos amparadores, o que está em avaliação ainda.

\begin{pullquote}
    ``Durante a escrita, me sentia muito motivada e energizada, o mentalsoma não cansava''
\end{pullquote}


\textbf{3.       Qual o maior aprendizado com a escrita desta obra?}

Aprendi muito sobre o processo de organização de uma obra para publicação. Uma coisa é escrever textos para nós mesmos ou trabalhos para cursos. Outra, muito diferente, é o livro. Muitas etapas são necessárias e muitas pessoas são envolvidas e trabalham arduamente junto com o autor. Durante o processo, tornei-me participante do Conselho Editorial da Editares e busco auxiliar outros autores nesse processo. É preciso persistência e paciência; a pressa em fazer rápido acaba assediando o fluxo e atrasando mais ainda os trabalhos. É preciso trabalhar ombro a ombro, tanto com os amparadores extrafísicos, como também com os colegas intrafísicos nas várias etapas de edição do livro. Entrar em conflito é contraproducente.

\begin{pullquote}
    ``É preciso persistência e paciência; a pressa em fazer rápido acaba assediando o fluxo e atrasando mais ainda os trabalhos.''
\end{pullquote}

\textbf{4.       O que poderia dizer como incentivo para que mais pesquisadores invistam na publicação de obras conscienciológicas?}

Escolha uma temática de interesse, busque bibliografia em todas as bases de consulta da Conscienciologia e comece a focar naquele assunto. Busque convergir as atividades para aquela temática, anote as experiências. Antes de começar a escrever, defina o escopo, a estrutura do livro e o público-alvo. E busque assessoria da Uniescon. Há várias atividades na IC para ajudar o autorando. 
    
    
    \end{multicols}
\end{document}
