\documentclass{gescons}

\genre {Entrevista}
\author{Karine Brito}
\title{Gratidão: Reconhecer, Expressar e Retribuir}

\begin{document}
    \makeentrevistatitle

    \begin{multicols}{2}


\noindent\includegraphics[width=9cm, height=10cm]{example-image} 

\textbf{1. Qual foi a motivação para a escrita da obra? Por que a definição deste tema para publicação de um livro?}

A motivação para a escrita do livro \textit{Gratidão: Reconhecer, Expressar e Retribuir} foi a descoberta de uma questão de saúde congênita em 2008, suscitando uma série de autorreflexões sobre a vida, os vínculos conscienciais nutridos até então e, sobretudo, o subnível evolutivo quanto à programação existencial.  A doença foi um alerta consciencial para ampliar o autodiscernimento, possibilitando identificar um grande travão evolutivo: a ingratidão. A escrita do livro surgiu como uma forma de retribuição lúcida das inúmeras benesses recebidas nesse processo. A inspiração extrafísica era explicitar como se dá o desenvolvimento da cognição sobre a gratidão, incentivando o seu cultivo autoconsciente no dia a dia. 

\begin{pullquote}
``A doença foi um alerta consciencial para ampliar o autodiscernimento, possibilitando identificar um grande travão evolutivo: a ingratidão.''
\end{pullquote}

\textbf{2. Quais foram as principais percepções, intra e extrafísicas, durante a pesquisa e a escrita da obra? E posterior ao lançamento?}

Ao longo dos 16 anos de pesquisa e escrita desta obra percebi de modo ostensivo a atuação dos amparadores, intra e extrafísicos. Muitos amigos evolutivos acompanharam de perto o desenvolvimento da pesquisa, ajudando nas autorreflexões, nos debates sobre o tema e nas revisões. Durante a escrita do livro, desde o início tive muitas inspirações da equipex. Percebia um interesse genuíno dos amparadores em materializar esta publicação, destacando a sua relevância no momento evolutivo atual. A principal inspiração extrafísica foi apresentar de modo técnico e didático a gratidão como base da recomposição grupocármica. Posterior ao lançamento, me chamou a atenção a assistência a para ouvintes de curso intermissivo ligados à invéxis, muito atentos ao tema da gratidão.

\begin{pullquote}
``Percebia um interesse genuíno dos amparadores em materializar esta publicação, destacando a sua relevância no momento evolutivo atual.''
\end{pullquote}

\textbf{3. Qual o maior aprendizado com a escrita desta obra?}

O maior aprendizado com a escrita desta obra foi, sem dúvida, levar a sério o alerta consciencial proexológico, colocando a recomposição grupocármica como prioridade. Isso foi a mola propulsora evolutiva para desenvolver a gratidão. 

\textbf{4. O que poderiam dizer como incentivo para que mais pesquisadores invistam na publicação de obras conscienciológicas?}

A obra conscienciológica é retribuição fundamental do intermissivista, favorecendo o autorrevezamento lúcido e a ampliação da interassistencialidade tarística a partir da recin do autor. Se você melhorou, ajude outras pessoas a chegarem até lá! 
    
    
    \end{multicols}
\end{document}



