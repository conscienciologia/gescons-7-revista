\documentclass{gescons}

\genre {Atualizações}
\author{Roberta Bouchardet}
\authorrole{Coordenadora da Escola de Editores Conscienciológicos} 
\title{Nova Escola de Editores Impulsiona Formação Editoriológica na Editares}

\begin{document}
    \makeentrevistatitle
    %\maketitle

    %\fullwidthimage{fields}{b}

    %\coverart{back/editorial}
    \coverart{../fundo-generico}
    \vspace*{\fill}
    
    
% \begin{center}
%     \includegraphics[width=\paperwidth]{articles/atualizacoes/fotos/escola-editores/escola-editores4.jpeg} 
% \end{center}

    % \begin{figure}
    %     % \includegraphics[width=\textwidth]{landscape}
    %     \includegraphics[width=\paperwidth]{articles/atualizacoes/fotos/escola-editores/escola-editores4.jpeg} 
    % \end{figure}

    \noindent\makebox[\linewidth]{%
    \includegraphics[width=0.9\paperwidth,height=0.5\paperheight,keepaspectratio,trim={0 50 200 170},clip]{articles/atualizacoes/fotos/escola-editores/escola-editores4.jpeg}%
    }


    \begin{multicols}{2}

%\section{Nova Escola de Editores impulsiona formação editoriológica na Editares}\label{nova-escola-de-editores-impulsiona-formauxe7uxe3o-editorioluxf3gica-na-editares}

Em julho de 2025, teve início a~\emph{Escola de Editores Conscienciológicos.} Trata-se de iniciativa inédita, objetivando formar e~qualificar voluntários do \emph{Conselho Editorial.} O~curso foi formulado pelos atuais conselheiros, ex-gestores da Editares, voluntários da UNICIN e~da CCCI com experiência no processo editorial. \textbf{A meta é~ampliar o~número de editores, qualificar as publicações, otimizar o~fluxo editorial e~ampliar a~interação entre os editores mais experientes com os novos.}

A \emph{Escola} responde à~crescente demanda de autores e~livros. A~proposta ganhou corpo com apoio da UNICIN, dentro do movimento de reestruturação institucional realizado em 2025. Com 11 módulos quinzenais aos domingos, em formato híbrido, o~curso na versão piloto terminou em dezembro deste ano, combinando teoria e~prática editorial.

% \vspace{-2\baselineskip}

A expectativa é~consolidar a~formação como etapa contínua da atuação editorial na CCCI, atraindo novos voluntários. A~turma piloto serviu de base para ajustes para novas edições ou formatos gravados.

Estamos empolgados com as possibilidades que esse projeto pode gerar!


\begin{center}
    \includegraphics[width=\columnwidth]{articles/atualizacoes/fotos/escola-editores/escola-editores1.jpeg} 
\end{center}


    \end{multicols}
    \vspace*{\fill}
\end{document}

