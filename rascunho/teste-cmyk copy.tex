\documentclass[a4paper,12pt]{article}

% --- Pacote PDF/X (gera PDF/X-1a:2001 com perfil CMYK) ---
\usepackage[a-1b]{pdfx}
\usepackage[x-1a,iccprofile=PSOcoated_v3.icc]{pdfx}


% --- Metadados obrigatórios do PDF/X ---
\begin{filecontents*}{\jobname.xmpdata}
\Title{Exemplo PDF/X em CMYK}
\Author{Eduardo Santana}
\Creator{LaTeX + pdfx}
\Publisher{Editares}
\Copyright{(c) 2025}
\Keywords{CMYK, PDF/X, impressão, LaTeX}
\end{filecontents*}

% --- Configuração de cores ---
\usepackage[dvipsnames]{xcolor}

% Define cores em CMYK (exemplo)
\definecolor{meuazul}{cmyk}{1,0.5,0,0}   % Azul em CMYK
\definecolor{meuvermelho}{cmyk}{0,1,1,0} % Vermelho em CMYK
\definecolor{meupreto}{cmyk}{0,0,0,1}    % Preto puro

\begin{document}

\section*{PDF em CMYK com perfil ICC embutido}

Este PDF foi gerado usando o pacote \texttt{pdfx}, que cria um arquivo
compatível com o padrão \textbf{PDF/X-1a:2001}, próprio para impressão gráfica.

\vspace{1cm}

\textcolor{meuazul}{\rule{10cm}{1cm}}\\[4pt]
\textcolor{meuvermelho}{\rule{10cm}{1cm}}\\[4pt]
\textcolor{meupreto}{\rule{10cm}{1cm}}

\vspace{1cm}
Verifique no Adobe Acrobat em:
\begin{quote}
Ferramentas → Impressão → Visualização de Saída
\end{quote}
para confirmar que as cores estão em CMYK.

\end{document}
