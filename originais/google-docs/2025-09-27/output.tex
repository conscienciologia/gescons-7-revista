CAPA

GESCONS

Editares celebra 20 anos com programação especial

Editares conquista a Certificação Institucional da UNICIN em 2024

Conscienciologia em Expansão: Distribuição de Publicações no Japão e Europa

Da Ideia do Autor às Mãos do Leitor: o Fluxo Editorial Conscienciológico

Nova Escola de Editores impulsiona formação editoriológica na Editares

Editares e IIPC retomam parceria para ampliar a venda de livro

Grupos de trabalho com a UNICIN traçam planos para fortalecer a atuação da Editares

GESCONS

REVISTA DA ASSOCIAÇÃO INTERNACIONAL EDITARES

Ano 6. Nº 7 -- Dezembro de 2025 -- Periodicidade Bienal

Coordenação Geral Editares 2024/2025: Ana Claudia Prado e Magda Amancio Stapf

Editora Geral: Amanda Vieira

\hl{Revisão:}

Diagramação: Eduardo Santana

\hl{Capa: Francielle Padilha}

Fotos: Acervo Editares

Tiragem: 300 exemplares

Impressão: Meta Brasil

\hl{ISNN: 2764-3468}

Histórico de publicações

1ª Edição: agosto/2014

2ª Edição: dezembro/2017

3ª Edição: dezembro/2019

4ª Edição: dezembro/2021

5ª Edição: novembro/2023

6ª Edição: dezembro/2023

EDITARES

Av. Felipe Wandscheer, 6200 -- Sala 110

Bairro Cognópolis -- Foz do Iguaçu -- PR

CEP 85856-850

Tel.: + 55 (45) 9 8843-8668

\href{http://www.editares.org}{\ul{www.editares.org}}

\href{mailto:contato@editares.org}{\ul{contato@editares.org}}

Facebook: editareseditora

Youtube: @editareseditora

Instagram: @editareseditora

REVISTA DA ASSOCIAÇÃO INTERNACIONAL EDITARES

EDIÇÃO N. 7 \textbar{} DEZEMBRO \textbar{} 2025

GESCONS

Editares 21 anos

\hl{EDITORIAL}

\textbf{20 anos de publicação interassistencial: passado, presente e futuro da Editares}

Em 2024, a Editares completou 20 anos de dedicação à publicação técnico-científica de obras conscienciológicas. Duas décadas de trabalho tarístico, construídas por muitas mãos, em que cada autor, revisor, diagramador, conselheiro, voluntário e leitor fez parte de um mesmo propósito: \textbf{expandir o esclarecimento e favorecer a evolução das consciências por meio da escrita}.

Celebrar essa história significa reconhecer o valor do esforço coletivo. Desde os primeiros livros publicados até os mais recentes lançamentos, cada obra representa um marco de interassistência, registrando aprendizados, reflexões e contribuições para o desenvolvimento da Conscienciologia e para a maxiproéxis grupal.

Esta edição da Revista Gescons propõe um olhar integrado sobre a trajetória da Editares: passado, presente e futuro se encontram para inspirar novos desafios e conquistas. É um convite para pensar no que já realizamos juntos e, principalmente, no que ainda podemos alcançar coletivamente.

Na primeira seção, apresentamos \textbf{resumo do biênio} com os acontecimentos mais relevantes da Editares entre 2023 e 2024: a conquista da Certificação Institucional da UNICIN, a participação no Congraçamento das ICs, a distribuição internacional de materiais no Japão e na Europa, as atualizações do fluxo editorial e os números que refletem nossa produção recente. Um destaque especial vai para a listagem completa dos voluntários atuais, organizados por equipes de trabalho, valorizando quem faz a Editares acontecer.

Na segunda seção, trazemos \textbf{atualizações} e novidades institucionais: a criação da Escola de Editores, voltada à formação e qualificação de novos voluntários; as novas estratégias de marketing; a nova revista científica da Editares para expansão da especialidade Editoriologia e Publicaciologia; a reativação da parceria com o IIPC para ampliar a venda de livros; e os grupos de trabalho com a UNICIN, responsáveis por repensar processos comerciais e editoriais.

Por fim, na terceira seção, apresentamos os \textbf{lançamentos de 2024 e 2025}, com obras que ampliam o debate técnico e aprofundam a pesquisa conscienciológica.

Mais do que celebrar o passado, esta edição propõe \textbf{um olhar para o futuro}. Queremos inspirar novos autores, atrair mais voluntários e consolidar práticas editoriais cada vez mais qualificadas e interassistenciais. Seguimos firmes no propósito de transformar ideias em livros, livros em esclarecimento e esclarecimento em evolução.

Agradecemos a todos os voluntários, autores, leitores e parceiros que fazem parte desta história. Que os próximos anos sejam ainda mais produtivos, lúcidos e interassistenciais.

Boa leitura!

Amanda Vieira\textbf{\hfill\break
} Editora desta Edição

{[}FOTO DA EDITORA{]}

SUMÁRIO

\textbf{Resumo do Biênio}

Gestão 2024-2025: Entrevista com as Coordenadoras

Editares celebra 20 anos com programação especial

Plaquinhas, sorrisos e escrita: Editares no Congraçamento das ICs

Editares conquista a Certificação Institucional da UNICIN em 2024

Conscienciologia em Expansão: Distribuição de Publicações no Japão e Europa

Da Ideia do Autor às Mãos do Leitor: o Fluxo Editorial Conscienciológico

Números do Biênio

Voluntários: Quem faz a Editares acontecer

\textbf{Atualizações}

Nova Escola de Editores impulsiona formação editoriológica na Editares

Projetos digitais modernizam a comunicação editorial da Editares

Publicação científica da Editares ganha espaço próprio a partir de 2025

Editares e IIPC retomam parceria para ampliar a venda de livro

Grupos de trabalho com a UNICIN traçam planos para fortalecer a atuação da Editares

\textbf{Entrevistas - Lançamentos 2024/2025}

RESUMO DO BIÊNIO

\textbf{Gestão 2024-2025: Entrevista com as Coordenadoras}

\textbf{Como foi o convite ou o processo de chegada de vocês à coordenação da Editares? Quais eram suas expectativas ao assumir e como elas se confirmaram ou mudaram nesses dois anos?}

\textbf{Ana:} Foi um convite bastante refletido. Precisei pensar muito antes de aceitar, porque já tinha diversas atividades. Houve um grande investimento das lideranças da Editares para que eu assumisse a coordenação, e uma das minhas condições foi que tivéssemos uma dupla. Então, convidamos a Magda.

\textbf{Magda:} Eu já tinha decidido que só toparia se fosse com a Ana. Se ela aceitasse, eu ficaria junto.

\textbf{Ana:} No início, eu não tinha muitas expectativas. As ideias foram surgindo com o tempo e fomos fazendo acontecer. Começamos organizando os contratos de cessão de direitos e planejando a mudança da sala, para acomodar melhor a equipe.

\textbf{Magda:} Eu tinha feito o curso de preparação de lideranças da Unicin, mas na época não compreendia bem o que representava. No curso, cheguei a dizer que não assumiria nenhuma gestão. Perguntaram do que eu estava me escondendo --- e essa pergunta ficou ecoando, me fazendo refletir.

\textbf{Ana:} E ainda por cima assumir a gestão da Editares, que tem uma atividade relacionada à toda CCCI. Eu também não pensava em coordenação, nem em voluntariar na Editares. Mas, ao publicar meu livro \emph{Antologia da Técnica de Mais um Ano de Vida}, senti vontade de ajudar mais, pois me identifiquei com o trabalho. Aos poucos, percebi o quanto isso se conecta com a minha proéxis: auxiliar outras pessoas a escreverem seus livros.

\textbf{Quais foram os maiores desafios enfrentados na gestão da editora nesse período? Que aprendizados pessoais e grupais vocês destacariam dessa experiência?}

\textbf{Ana:} Um dos maiores desafios foi formar equipes e reestruturar o conselho. Outro foi dar mais celeridade às publicações. Isso só foi possível graças ao envolvimento dos voluntários, autores, editores e da própria CCCI. Mostramos do que somos capazes quando trabalhamos juntos. Conseguimos organizar o fluxo de chegada das obras, fazer o acompanhamento até a publicação e manter um bom ritmo. Essa ``máquina'' só funciona porque contamos com especialistas para revisão, conferência e apoio editorial. Hoje, tudo flui bem graças ao esforço coletivo.

\textbf{Magda:} Essa experiência exigiu um amadurecimento na gestão, porque tivemos que desenvolver uma visão de conjunto. Para mim, um desafio marcante foi a quebra do sistema de gestão de estoque. Perdemos dados atualizados e tivemos que fazer um esforço enorme para recuperá-los. Trabalhei lado a lado com a Angélica {[}funcionária da Editares{]} nessa fase. A Ana estava viajando, e isso acabou ajudando, porque ela conseguia me acalmar e me manter focada na solução. Essa parceria na coordenação funcionou muito bem: quando uma estava sobrecarregada, a outra dava suporte.

\textbf{Que efeitos essa experiência trouxe para a autopesquisa e a proéxis de cada uma?}

\textbf{Ana:} Para mim, foi como se eu tivesse me localizado na proéxis. É como se tivesse feito um ``download'' não só da intermissão, mas também de experiências de outras vidas. Estar na linha de frente, na ``vitrine'', não era algo natural para mim. Aprendi a lidar com a exposição, a assumir responsabilidades e a mostrar quem eu realmente sou --- com meus traf\emph{o}res, traf\emph{a}res e traf\emph{ai}s. Um dos maiores aprendizados foi não limitar meus potenciais por medo de errar ou aparecer. Essa vivência me ajudou a superar essas barreiras internas.

\textbf{Magda:} Para mim, a gestão ampliou muito minha visão. Passei a enxergar melhor o funcionamento do grupo, amadureci e desenvolvi um senso maior de pertencimento à Comunidade Conscienciológica. Essa sensação de estar integrada e contribuir para algo maior é muito gratificante.

\textbf{Como vocês imaginam o futuro da editoração conscienciológica?}

\textbf{Ana:} Gostaria que conseguíssemos dar ainda mais celeridade às publicações. Para isso, precisamos que as obras cheguem mais bem estruturadas, o que permitiria avançar com mais rapidez. Estamos trabalhando nesse sentido.

\textbf{Magda:} As parcerias com a Uniescon podem contribuir bastante nesse processo.

\textbf{Ana:} Além disso, seria importante ampliar o engajamento de voluntários da CCCI, especialmente aqueles com disponibilidade para atuar nas revisões.

\textbf{Que mensagem final gostariam de deixar para os leitores e voluntários da Conscienciologia?}

\textbf{Ana:} Se uma oportunidade ou desafio bater à sua porta, abrace-o e leve até o fim. No final, reflita: \emph{o que aprendi com isso?}

\textbf{Magda:} Apesar dos desafios da coordenação, esse trabalho vale muito a pena. Ele contribui diretamente para o amadurecimento pessoal, grupal e intraconsciencial.

\textbf{Coordenadoras:} Agradecemos a todos os voluntários da Editares e da CCCI que contribuem e fazem parte desse trabalho interassistencial de publicação de obras conscienciológicas, ajudando a materializar as proéxis individuais e a maxiproéxis grupal!

{[}FOTO DAS COORDENADORAS{]}

RESUMO DO BIÊNIO

\textbf{Editares celebra 20 anos com programação especial}

No dia 23 de outubro de 2024, a Editares completou duas décadas de atuação dedicada à publicação técnico-científica de obras conscienciológicas. Para marcar os 20 anos de existência, a editora promoveu uma série de atividades comemorativas ao longo da semana, integrando voluntários, autores e leitores em momentos de reflexão, confraternização e interassistência.

A programação contou com uma \textbf{semana de verbetes temáticos}, realizada no \emph{Tertuliarium} do CEAEC, com a participação de autores e leitores em debates e apresentações de temas relacionados ao propósito editorial da Editares. Também houve um \textbf{Círculo Mentalsomático temático}, focado na escrita interassistencial e na importância das publicações para a expansão da Conscienciologia.

Um dos pontos altos da celebração foi o \textbf{lançamento do livro ``Interassistência''}, de autoria da voluntária do Conselho Editorial, \textbf{Roberta Bouchardet}. A obra amplia a abordagem da assistência multidimensional.

Além das atividades de debate, os voluntários participaram de um \textbf{jantar temático português}, organizado pelo restaurante Primener, no campus do CEAEC. A noite foi marcada pelo clima de gratidão e união entre os voluntários da Editares e convidados.

No dia 23, data oficial do aniversário, a comemoração seguiu após a Tertúlia, com \textbf{bolo comemorativo e momentos de confraternização} entre voluntários, amigos e apoiadores da Editares, celebrando o continuísmo do trabalho em equipe e o compromisso com a tarefa esclarecedora.

Com 20 anos de história, a Editares reafirma seu papel na difusão da Conscienciologia por meio da escrita e da publicação de obras que promovem o esclarecimento e a evolução das consciências. A celebração reforçou o valor da escrita interassistencial como instrumento de mudança e continuidade da maxiproéxis grupal.

{[}COLOCAR ALGUMAS FOTOS{]} As fotos estão na pasta do drive: Matéria 1: Editares celebra 20 anos com programação especial - elas podem ser num tamanho pequeno; aparecer no final ou intercalando com o texto

RESUMO DO BIÊNIO

\textbf{Plaquinhas, sorrisos e escrita: Editares no Congraçamento das ICs}

No dia 8 de dezembro de 2024, durante o tradicional \textbf{Congraçamento das ICs} -- evento anual de celebração do voluntariado conscienciológico --, a Editares marcou presença com uma ação voltada aos \textbf{autores e futuros autores}.

Com placas instigativas e frases de compromisso com a escrita, os voluntários foram convidados a tirar fotos simbolizando o engajamento na tarefa autoral. A atividade trouxe um clima de \textbf{descontração e leveza}, ao mesmo tempo em que reforçou a importância do autoposicionamento proexológico perante a escrita conscienciológica.

\textbf{Confira as fotos!}

{[}MURAL/MOSAICO DE FOTOS{]} Fotos na pasta matéria 2\ldots{} as fotos podem ocupar 2 páginas

RESUMO DO BIÊNIO

{[}MURAL/MOSAICO DE FOTOS{]}

RESUMO DO BIÊNIO

\textbf{Editares conquista a Certificação Institucional da UNICIN em 2024}

A Editares recebeu, pela segunda vez, a \emph{Certificação Institucional da UNICIN,} após avaliação realizada em 2024. A primeira certificação ocorreu em 2019. Esse reconhecimento reforça o alinhamento da IC à maxiproéxis grupal da \emph{Comunidade Conscienciológica Cosmoética Internacional} (CCCI) e ao compromisso contínuo com a interassistencialidade tarística.

A \emph{Certificação Institucional} promovida pela UNICIN não tem caráter técnico ou legal como as certificações normativas nacionais, mas representa um \emph{aval grupal,} validando a atuação funcional e interassistencial da IC dentro do fluxo da CCCI.

O processo é coordenado pelo \emph{Comitê Conscienciocêntrico da UNICIN,} que envia um formulário-padrão às \emph{Instituições Conscienciocêntricas} (ICs). As respostas são analisadas por conselhos especializados:

\begin{itemize}
\item
  Conselho de Interassistência Jurídica da Conscienciologia (CIAJUC)
\item
  Conselho Intercientífico
\item
  Conselho de Parapedagogia
\item
  Conselho Intervoluntariado
\item
  Conselho de Interassistência em Economia, Finanças e Orientação Fiscal (CIEFFI)
\end{itemize}

A certificação visa verificar o grau de maturidade institucional em aspectos como gestão, interassistência, parapedagogia, voluntariado e sustentabilidade organizacional, sempre sob a ótica dos princípios conscienciais.

A conquista da certificação pela Editares em 2024 reflete o esforço coletivo dos voluntários e a consolidação do trabalho editorial tarístico, contribuindo com a reurbex planetária e a expansão da Conscienciologia.

Editares agradece à UNICIN e aos conselhos envolvidos pela seriedade no processo, e renova seu compromisso com a interassistência tarística e o fortalecimento da maxiproéxis grupal.

\textbf{Ana Claudia Prado}

\textbf{Magda Stapf Amancio\\
}\emph{Coordenadoras Editares -- Biênio 2024--2025}

\textbf{{[}INSERIR UMA DAS FOTOS DA PASTA ``MATÉRIA 3''{]}}

RESUMO DO BIÊNIO

\textbf{Conscienciologia em Expansão: Distribuição de Publicações no\\
Japão e Europa}

\emph{\textbf{Expedição Paracientífica para a Expo 2025 Osaka viabiliza publicação do primeiro suplemento da Revista Holotecologia em inglês e lança miniglossário conscienciológico inglês-japonês.}}

\emph{Por Luziânia Medeiros e Paulo Abrantes}

\emph{Coordenadores da Expedição Paracientífica Expo 2025 Osaka}

A \emph{Expedição Paracientífica Expo 2025 Osaka} consistiu numa viagem técnica grupal, planejada em detalhes durante 2 anos, incluindo instrumentos de registro e documentação para coleta, sistematização, discussão, aprofundamento e síntese dos achados e parachados. Tudo isso visando pensar o \emph{Megacentro Cultural Holoteca} em ambiente universalista, pacífico e multicultural de uma Expo Mundial. Ao mesmo tempo, considerando o tema dessa Expo 2025 em Osaka, \emph{projetando a sociedade do futuro para nossas vidas,} objetivamos \textbf{contribuir com o conceito de Cognópolis como nosso modelo inovador de sociedade do futuro}.

\textbf{GESTAÇÃO DE PUBLICAÇÕES}

Em relação às publicações, a revista Holotecologia, publicada a cada 2 anos, já está NA 5ª Edição, na qual escrevemos uma matéria especial sobre as Exposições Universais em que fizemos um cotejo desses megaeventos mundiais com o futuro MCH. Uma ideia inicial para fazermos a interlocução com o público da Expo era a de traduzir esse artigo para o inglês. Porém, observou-se a necessidade de apresentar a Cognópolis, a neociência Conscienciologia, seu propositor e pesquisadores, e mostrar os principais projetos, dentre eles o Jovem Pesquisador, selecionado no programa de Co-criação da Expo 2025 Osaka e o projeto do Megacentro Cultural Holoteca, além de artigo da Efemeroteca exemplificando as coleções. Daí nasceu o projeto \emph{Holothecology Supplemen}t, \textbf{primeira versão} da revista em inglês.

Por outro lado, estávamos indo para o Japão, e parte da equipe da Expedição estava há 1 ano e meio trabalhando no projeto do dicionário de termos da Conscienciologia em japonês. Faltando pouco mais de 2 meses para a viagem sugerimos à equipe avaliar a possibilidade de publicar a primeira edição gratuita com termos produzidos até o momento, em formato de miniglossário, para ser distribuído entre os japoneses, presenteando-os com neovocábulos no idioma local. A equipe da Nipoteca da Holoteca aceitou prontamente o desafio e \textbf{fez a obra acontecer em 2 meses.}

\textbf{A FORÇA DO GRUPO: FINANCIAMENTO E DISTRIBUIÇÃO DAS PUBLICAÇÕES}

A produção dessas publicações estava a todo vapor, porém havia um desafio: como financiar as impressões. Daí surgiu a ideia de captação por meio de um item colecionável, uma edição limitada da moeda comemorativa da Expedição. As doações ganharam força e o recurso arrecadado possibilitou a impressão de \textbf{1000 revistas e 500 mini-glossários.} Essa realização foi grupal, voluntária e com o propósito de divulgar as ideias da Conscienciologia para o público internacional mais amplo.

O material chegou há tempo de ser levado por alguns pesquisadores que estavam de viagem para vários países da Europa, em curso itinerante do \emph{Cosmovisão} promovido pela Assinvéxis. Foram disponibilizados dezenas de exemplares da revista \emph{Holothecology,} distribuídos em países como Inglaterra, França, Áustria e Suíça. Segundo relatos, a revista ajudou a criar pontes de interlocução em diferentes contextos ao longo da viagem.

O transporte das obras para o Japão foi possível devido à colaboração dos integrantes da Expedição, que disponibilizaram espaço em suas bagagens para levar 1 kit padrão de publicações composto por 10 revistas, 10 miniglossários e 4 \emph{Our Evolution}. Além disso, levamos conjuntos do \emph{Léxico de Ortopensatas} e o \emph{Manual dos Megapensenes Trivocabulares}. \textbf{A união faz a força!}

\textbf{BALANÇO E PROSPECTIVAS}

Distribuímos no Japão 260 \emph{Miniglossários}, 215 \emph{Holothecology}, 36 \emph{Our Evolution}, 2 \emph{Léxicos de Ortopensatas} e 1 \emph{Manual dos Megapensenes Trivocabulares}. As publicações foram deixadas em 8 cidades do Japão, incluindo bibliotecas, sebos, universidades e instituições, e na Expo circularam em pelo menos 43 pavilhões de países dos 5 continentes, e em pavilhões de instituições internacionais como ONU, União Europeia, Cruz Vermelha e Bureau International des Expositions (BIE). Sem falar dos contatos espontâneos com japoneses em vários locais visitados no Japão e com pessoas de várias nacionalidades que visitavam a Expo.

Foram estabelecidos vários canais de contatos com pessoas e instituições, verdadeiras \textbf{pontes interassistenciais} de natureza variada a serem cultivadas e desenvolvidas ao longo dos próximos anos.

Percebemos a importância e relevância do miniglossário inglês-japonês também para os intermissivistas recém-ressomados e os que irão ressomar num futuro próximo naquele país, onde ainda não há atividades de Conscienciologia, ante um passado conflituoso e um posicionamento firme pela paz a partir do pós-guerra.

A revista, por sua vez, plantou a semente do modelo de sociedade que vem se desenvolvendo a partir da Cognópolis Foz do Iguaçu na região da trifron e da própria CCCI, fortalecendo a Conscienciologia ao redor do mundo.

Essas iniciativas fortalecem o avanço do Megacentro Cultural Holoteca, um megaempreendimento grupal prioritário, de natureza policármica, abrangente, e que contribuirá para a consolidação da Conscienciologia no planeta.

\textbf{{[}INSERIR FOTOS DA PASTA MATÉRIA 4 - \emph{falta receber essas fotos}{]}}

RESUMO DO BIÊNIO

\textbf{Da Ideia do Autor às Mãos do Leitor: o Fluxo Editorial Conscienciológico}

\emph{\textbf{Processo editorial une rigor, voluntariado e compromisso para lançar obras conscienciológicas}}

\emph{Por Ana Claudia Prado e Magda Stapf}

\emph{Coordenação Geral da Editares}

Publicar um livro não é apenas escrever. Na Editares, cada obra passa por um processo integrado que alia rigor editorial, trabalho voluntário e compromisso interassistencial tarístico para \textbf{garantir qualidade e relevância} antes de chegar ao leitor.

O fluxo editorial começa com o envio dos originais textuais pelo autor para o e-mail da Editares. Em seguida, o material é organizado na plataforma \emph{Trello Editorial} e submetido à pré-análise por uma equipe voluntária, que analisa conteúdo, estrutura e a forma. Quando necessário, o texto retorna ao autor para melhorias até estar pronto para avançar.

Com a aprovação da pré-análise, o material textual ingressa no fluxo formal de produção editorial, que contempla parecer técnico, assinatura do termo de cessão de direitos autorais, e revisões linguístico-textual e de conteúdo e forma --- conhecidas como \emph{Confor.} Durante esta etapa, o editor e revisores especializados trabalham em parceria com o autor, garantindo autenticidade e qualidade técnica.

Com o texto finalizado, o livro passa para a diagramação, momento em que o conteúdo ganha forma gráfica, tornando-se fluido e agradável à leitura. O autor pode optar pela diagramação voluntária da Editares ou contratar um profissional externo. Antes de seguir para a impressão, uma prova física é cuidadosamente revisada para assegurar o acabamento.

Após a definição do orçamento e da tiragem, a obra é encaminhada para a gráfica. Paralelamente, a equipe de comunicação prepara o lançamento do livro, um \textbf{momento de celebração da conquista do autor} e de início da circulação da obra entre leitores em geral.

Mais do que um produto editorial, cada título publicado pela Editares representa a \textbf{difusão da Conscienciologia,} fortalecendo a interassistência e promovendo o crescimento evolutivo de todos os envolvidos.

\textbf{FLUXO EDITORIAL DE OBRA CONSCIENCIOLÓGICA}

\emph{\textbf{\ul{(solicitar imagem em alta resolução à Ana Cláudia)}}}

\includegraphics[width=5.02083in,height=3.11458in]{media/image1.png}

RESUMO DO BIÊNIO

\textbf{Números do Biênio}

\textbf{\hl{LANÇAMENTOS DE OBRAS EM PORTUGUÊS}}

\begin{longtable}[]{@{}
  >{\raggedright\arraybackslash}p{(\linewidth - 4\tabcolsep) * \real{0.5327}}
  >{\raggedright\arraybackslash}p{(\linewidth - 4\tabcolsep) * \real{0.2804}}
  >{\raggedright\arraybackslash}p{(\linewidth - 4\tabcolsep) * \real{0.1869}}@{}}
\toprule\noalign{}
\begin{minipage}[b]{\linewidth}\centering
\textbf{Título}
\end{minipage} & \begin{minipage}[b]{\linewidth}\centering
\textbf{Autor}
\end{minipage} & \begin{minipage}[b]{\linewidth}\centering
\textbf{Lançamento}
\end{minipage} \\
\begin{minipage}[b]{\linewidth}\raggedright
\textbf{Autolegado Evolutivo}
\end{minipage} & \begin{minipage}[b]{\linewidth}\raggedright
\textbf{Haydée Melo}
\end{minipage} & \begin{minipage}[b]{\linewidth}\raggedright
06/04/2024
\end{minipage} \\
\begin{minipage}[b]{\linewidth}\raggedright
\textbf{Mutatis Mutandis: Teoria e Prática da Reciclagem Existencial}
\end{minipage} & \begin{minipage}[b]{\linewidth}\raggedright
\textbf{Luimara Schmit}
\end{minipage} & \begin{minipage}[b]{\linewidth}\raggedright
10/08/2024
\end{minipage} \\
\begin{minipage}[b]{\linewidth}\raggedright
\textbf{Megapensenes Trivocabulares Pró-evolutivos}
\end{minipage} & \begin{minipage}[b]{\linewidth}\raggedright
\textbf{Michel Chad}
\end{minipage} & \begin{minipage}[b]{\linewidth}\raggedright
17/08/2024
\end{minipage} \\
\begin{minipage}[b]{\linewidth}\raggedright
\textbf{Epicentrismo Consciencial: Casuísticas Recinológicas}
\end{minipage} & \begin{minipage}[b]{\linewidth}\raggedright
\textbf{Ana Luiza Rezende e Mabel Teles}
\end{minipage} & \begin{minipage}[b]{\linewidth}\raggedright
24/08/2024
\end{minipage} \\
\begin{minipage}[b]{\linewidth}\raggedright
\textbf{Interassistência: Teoria e Prática sob a Ótica da Conscienciologia}
\end{minipage} & \begin{minipage}[b]{\linewidth}\raggedright
\textbf{Roberta Bouchardet}
\end{minipage} & \begin{minipage}[b]{\linewidth}\raggedright
19/10/2024
\end{minipage} \\
\begin{minipage}[b]{\linewidth}\raggedright
\textbf{Gratidão: Reconhecer, Expressar e Retribuir}
\end{minipage} & \begin{minipage}[b]{\linewidth}\raggedright
\textbf{Karine Brito}
\end{minipage} & \begin{minipage}[b]{\linewidth}\raggedright
16/11/2024
\end{minipage} \\
\begin{minipage}[b]{\linewidth}\raggedright
\textbf{Autoexperimentação Conscienciológica}
\end{minipage} & \begin{minipage}[b]{\linewidth}\raggedright
\textbf{Alexandre Zaslavsky}
\end{minipage} & \begin{minipage}[b]{\linewidth}\raggedright
23/11/2024
\end{minipage} \\
\begin{minipage}[b]{\linewidth}\raggedright
\textbf{O Calidoscópio da evolução consciencial}
\end{minipage} & \begin{minipage}[b]{\linewidth}\raggedright
\textbf{Maria Cristina Bassanesi}
\end{minipage} & \begin{minipage}[b]{\linewidth}\raggedright
30/11/2024
\end{minipage} \\
\begin{minipage}[b]{\linewidth}\raggedright
\textbf{Ortoprincípios Norteadores da Invéxis}
\end{minipage} & \begin{minipage}[b]{\linewidth}\raggedright
\textbf{Ricardo Rezende}
\end{minipage} & \begin{minipage}[b]{\linewidth}\raggedright
07/10/2024
\end{minipage} \\
\begin{minipage}[b]{\linewidth}\raggedright
\textbf{Ações Antifanatismo}
\end{minipage} & \begin{minipage}[b]{\linewidth}\raggedright
\textbf{Daniel Mamede}
\end{minipage} & \begin{minipage}[b]{\linewidth}\raggedright
26/04/2025
\end{minipage} \\
\begin{minipage}[b]{\linewidth}\raggedright
\textbf{Autoidentificação Proexológica}
\end{minipage} & \begin{minipage}[b]{\linewidth}\raggedright
\textbf{Ricardo Rezende}
\end{minipage} & \begin{minipage}[b]{\linewidth}\raggedright
28/06/2025
\end{minipage} \\
\begin{minipage}[b]{\linewidth}\raggedright
\textbf{Eitologia: Teática Inversiva}
\end{minipage} & \begin{minipage}[b]{\linewidth}\raggedright
\textbf{Pedro Borges}
\end{minipage} & \begin{minipage}[b]{\linewidth}\raggedright
12/07/2025
\end{minipage} \\
\begin{minipage}[b]{\linewidth}\raggedright
\textbf{Retomada de Tarefa: Reencontro com o Paradever Intermissivo}
\end{minipage} & \begin{minipage}[b]{\linewidth}\raggedright
\textbf{Ana Mazzonetto}
\end{minipage} & \begin{minipage}[b]{\linewidth}\raggedright
19/07/2025
\end{minipage} \\
\begin{minipage}[b]{\linewidth}\raggedright
\textbf{Autoinversão Existencial: Compreensão e Vivência}
\end{minipage} & \begin{minipage}[b]{\linewidth}\raggedright
\textbf{Paula Gabriella Barbosa}
\end{minipage} & \begin{minipage}[b]{\linewidth}\raggedright
26/07/2025
\end{minipage} \\
\begin{minipage}[b]{\linewidth}\raggedright
\textbf{Estado Vibracional: 100 Perguntas e Respostas}
\end{minipage} & \begin{minipage}[b]{\linewidth}\raggedright
\textbf{Victor Bolfe}
\end{minipage} & \begin{minipage}[b]{\linewidth}\raggedright
02/08/2025
\end{minipage} \\
\begin{minipage}[b]{\linewidth}\raggedright
\textbf{Migrações Proexológicas Internacionais}
\end{minipage} & \begin{minipage}[b]{\linewidth}\raggedright
\textbf{Virgínia Ruiz}
\end{minipage} & \begin{minipage}[b]{\linewidth}\raggedright
30/08/2025
\end{minipage} \\
\begin{minipage}[b]{\linewidth}\raggedright
\textbf{Autoconscientização Grafopensênica}
\end{minipage} & \begin{minipage}[b]{\linewidth}\raggedright
\textbf{Denise Paro}
\end{minipage} & \begin{minipage}[b]{\linewidth}\raggedright
07/09/2025
\end{minipage} \\
\begin{minipage}[b]{\linewidth}\raggedright
\textbf{Tanatofobia: da Ressignificação da Existência à Superação do\,Medo\,da\,Morte}
\end{minipage} & \begin{minipage}[b]{\linewidth}\raggedright
\textbf{Walderley Carvalho}
\end{minipage} & \begin{minipage}[b]{\linewidth}\raggedright
14/09/2025
\end{minipage} \\
\begin{minipage}[b]{\linewidth}\raggedright
\textbf{Diário de um Iniciante na Tenepes}
\end{minipage} & \begin{minipage}[b]{\linewidth}\raggedright
\textbf{Ghunter Kismann}
\end{minipage} & \begin{minipage}[b]{\linewidth}\raggedright
20/09/2025
\end{minipage} \\
\begin{minipage}[b]{\linewidth}\raggedright
\textbf{Imobilidade Física Vígil: Autopesquisa Experimentológica}
\end{minipage} & \begin{minipage}[b]{\linewidth}\raggedright
\textbf{Fátima Fernandes}
\end{minipage} & \begin{minipage}[b]{\linewidth}\raggedright
27/09/2025
\end{minipage} \\
\begin{minipage}[b]{\linewidth}\raggedright
\textbf{Seleção Consciencial}
\end{minipage} & \begin{minipage}[b]{\linewidth}\raggedright
\textbf{Roberto Kunz}
\end{minipage} & \begin{minipage}[b]{\linewidth}\raggedright
04/10/2025
\end{minipage} \\
\begin{minipage}[b]{\linewidth}\raggedright
\textbf{Libertações Evolutivas: Trajetória Semperaprendente Sob Olhar Conscienciológico}
\end{minipage} & \begin{minipage}[b]{\linewidth}\raggedright
\textbf{Helena Araújo}
\end{minipage} & \begin{minipage}[b]{\linewidth}\raggedright
11/10/2025
\end{minipage} \\
\begin{minipage}[b]{\linewidth}\raggedright
\textbf{Da Labilidade Parapsíquica A Assistencialidade Lúcida: Uma Trajetória Evolutiva}
\end{minipage} & \begin{minipage}[b]{\linewidth}\raggedright
\textbf{Luis Claudio Resende}
\end{minipage} & \begin{minipage}[b]{\linewidth}\raggedright
08/11/2025
\end{minipage} \\
\begin{minipage}[b]{\linewidth}\raggedright
\textbf{A Lógica do Pensene}
\end{minipage} & \begin{minipage}[b]{\linewidth}\raggedright
\textbf{Isabela Collares}
\end{minipage} & \begin{minipage}[b]{\linewidth}\raggedright
15/11/2025
\end{minipage} \\
\begin{minipage}[b]{\linewidth}\raggedright
\textbf{Coadjuvantes da Invéxis}
\end{minipage} & \begin{minipage}[b]{\linewidth}\raggedright
\textbf{Felipe Oliveira}
\end{minipage} & \begin{minipage}[b]{\linewidth}\raggedright
22/11/2025
\end{minipage} \\
\begin{minipage}[b]{\linewidth}\raggedright
\textbf{Dicionário de Paradireito: Exemplarium}
\end{minipage} & \begin{minipage}[b]{\linewidth}\raggedright
\textbf{Marlene Roque}
\end{minipage} & \begin{minipage}[b]{\linewidth}\raggedright
06/12/2025
\end{minipage} \\
\begin{minipage}[b]{\linewidth}\raggedright
\textbf{TOTAL}
\end{minipage} & \begin{minipage}[b]{\linewidth}\raggedright
\textbf{2024: 9 títulos}
\end{minipage} & \begin{minipage}[b]{\linewidth}\raggedright
\textbf{2025: 17}
\end{minipage} \\
\midrule\noalign{}
\endhead
\bottomrule\noalign{}
\endlastfoot
\end{longtable}

\textbf{\hl{REEDIÇÕES DE OBRAS}}

\begin{longtable}[]{@{}
  >{\raggedright\arraybackslash}p{(\linewidth - 4\tabcolsep) * \real{0.5966}}
  >{\raggedright\arraybackslash}p{(\linewidth - 4\tabcolsep) * \real{0.2165}}
  >{\raggedright\arraybackslash}p{(\linewidth - 4\tabcolsep) * \real{0.1869}}@{}}
\toprule\noalign{}
\begin{minipage}[b]{\linewidth}\centering
\textbf{Título}
\end{minipage} & \begin{minipage}[b]{\linewidth}\centering
\textbf{Autor}
\end{minipage} & \begin{minipage}[b]{\linewidth}\centering
\textbf{Edição}
\end{minipage} \\
\begin{minipage}[b]{\linewidth}\raggedright
\textbf{Estado Vibracional: Vivências e Autoqualificação}
\end{minipage} & \begin{minipage}[b]{\linewidth}\raggedright
\textbf{Victor Bolfe}
\end{minipage} & \begin{minipage}[b]{\linewidth}\raggedright
\textbf{2ª Edição}
\end{minipage} \\
\begin{minipage}[b]{\linewidth}\raggedright
\textbf{Conscienciograma}
\end{minipage} & \begin{minipage}[b]{\linewidth}\raggedright
\textbf{Waldo Vieira}
\end{minipage} & \begin{minipage}[b]{\linewidth}\raggedright
\textbf{2ª Edição}
\end{minipage} \\
\begin{minipage}[b]{\linewidth}\raggedright
\textbf{Profilaxia das Manipulações Conscienciais}
\end{minipage} & \begin{minipage}[b]{\linewidth}\raggedright
\textbf{Mabel Teles}
\end{minipage} & \begin{minipage}[b]{\linewidth}\raggedright
\textbf{3ª Edição}
\end{minipage} \\
\begin{minipage}[b]{\linewidth}\raggedright
\textbf{Manual de Publicações da Editares}
\end{minipage} & \begin{minipage}[b]{\linewidth}\raggedright
\textbf{Lane Galdino}
\end{minipage} & \begin{minipage}[b]{\linewidth}\raggedright
\textbf{2ª Edição}
\end{minipage} \\
\begin{minipage}[b]{\linewidth}\raggedright
\textbf{Higiene Consciencial: Reconquistando a Homeostase no microuniverso}
\end{minipage} & \begin{minipage}[b]{\linewidth}\raggedright
\textbf{Eduardo Martins}
\end{minipage} & \begin{minipage}[b]{\linewidth}\raggedright
\textbf{3ª Edição}
\end{minipage} \\
\begin{minipage}[b]{\linewidth}\raggedright
\textbf{Mentalsomaticidade Evolutiva: Estudos Iniciais}
\end{minipage} & \begin{minipage}[b]{\linewidth}\raggedright
\textbf{Ricardo Rezende}
\end{minipage} & \begin{minipage}[b]{\linewidth}\raggedright
\textbf{2ª Edição}
\end{minipage} \\
\begin{minipage}[b]{\linewidth}\raggedright
\textbf{Voluntariado Conscienciológico Interassistencial}
\end{minipage} & \begin{minipage}[b]{\linewidth}\raggedright
\textbf{Ricardo Rezende}
\end{minipage} & \begin{minipage}[b]{\linewidth}\raggedright
\textbf{2ª Edição}
\end{minipage} \\
\begin{minipage}[b]{\linewidth}\raggedright
\textbf{Sem Medo da Morte: Construindo uma Realidade Multidimensional}
\end{minipage} & \begin{minipage}[b]{\linewidth}\raggedright
\textbf{Vera Hoffmann}
\end{minipage} & \begin{minipage}[b]{\linewidth}\raggedright
\textbf{2ª Edição}
\end{minipage} \\
\begin{minipage}[b]{\linewidth}\raggedright
\textbf{Qualificações da Consciência}
\end{minipage} & \begin{minipage}[b]{\linewidth}\raggedright
\textbf{Júlio Almeida}
\end{minipage} & \begin{minipage}[b]{\linewidth}\raggedright
\textbf{2ª Edição}
\end{minipage} \\
\begin{minipage}[b]{\linewidth}\raggedright
\textbf{TOTAL}
\end{minipage} & \begin{minipage}[b]{\linewidth}\raggedright
\textbf{9}
\end{minipage} & \begin{minipage}[b]{\linewidth}\raggedright
\end{minipage} \\
\midrule\noalign{}
\endhead
\bottomrule\noalign{}
\endlastfoot
\end{longtable}

\textbf{\hl{LANÇAMENTOS DE OBRAS IMPRESSAS EM OUTROS IDIOMAS}}

\begin{longtable}[]{@{}
  >{\raggedright\arraybackslash}p{(\linewidth - 4\tabcolsep) * \real{0.5966}}
  >{\raggedright\arraybackslash}p{(\linewidth - 4\tabcolsep) * \real{0.2165}}
  >{\raggedright\arraybackslash}p{(\linewidth - 4\tabcolsep) * \real{0.1869}}@{}}
\toprule\noalign{}
\begin{minipage}[b]{\linewidth}\centering
\textbf{Título}
\end{minipage} & \begin{minipage}[b]{\linewidth}\centering
\textbf{Autor}
\end{minipage} & \begin{minipage}[b]{\linewidth}\centering
\textbf{Idioma}
\end{minipage} \\
\begin{minipage}[b]{\linewidth}\raggedright
\textbf{Der Kleine Forscher - Multidimensionalität}
\end{minipage} & \begin{minipage}[b]{\linewidth}\raggedright
\textbf{Débora Klippel}
\end{minipage} & \begin{minipage}[b]{\linewidth}\raggedright
Alemão
\end{minipage} \\
\begin{minipage}[b]{\linewidth}\raggedright
\textbf{Glossário Árabe -- Inglês}
\end{minipage} & \begin{minipage}[b]{\linewidth}\raggedright
\textbf{Orgs. Mohammed Alamassi e Fátima Yahya}
\end{minipage} & \begin{minipage}[b]{\linewidth}\raggedright
Árabe
\end{minipage} \\
\begin{minipage}[b]{\linewidth}\raggedright
\textbf{Nossa Evolução}
\end{minipage} & \begin{minipage}[b]{\linewidth}\raggedright
\textbf{Waldo Vieira}
\end{minipage} & \begin{minipage}[b]{\linewidth}\raggedright
Árabe
\end{minipage} \\
\begin{minipage}[b]{\linewidth}\raggedright
\textbf{La Vida Sigue}
\end{minipage} & \begin{minipage}[b]{\linewidth}\raggedright
\textbf{Betânia Abreu}
\end{minipage} & \begin{minipage}[b]{\linewidth}\raggedright
Espanhol
\end{minipage} \\
\begin{minipage}[b]{\linewidth}\raggedright
\textbf{Síndrome de la Dispersión Conciencial}
\end{minipage} & \begin{minipage}[b]{\linewidth}\raggedright
\textbf{Neida Cardoso}
\end{minipage} & \begin{minipage}[b]{\linewidth}\raggedright
Espanhol
\end{minipage} \\
\begin{minipage}[b]{\linewidth}\raggedright
\textbf{Prescripciones para el autodesasedio}
\end{minipage} & \begin{minipage}[b]{\linewidth}\raggedright
\textbf{Maxmiliano Haymann}
\end{minipage} & \begin{minipage}[b]{\linewidth}\raggedright
Espanhol
\end{minipage} \\
\begin{minipage}[b]{\linewidth}\raggedright
\textbf{Nuestra Evolución}
\end{minipage} & \begin{minipage}[b]{\linewidth}\raggedright
\textbf{Waldo Vieira}
\end{minipage} & \begin{minipage}[b]{\linewidth}\raggedright
Espanhol
\end{minipage} \\
\begin{minipage}[b]{\linewidth}\raggedright
\textbf{Que és la Concienciología}
\end{minipage} & \begin{minipage}[b]{\linewidth}\raggedright
\textbf{Waldo Vieira}
\end{minipage} & \begin{minipage}[b]{\linewidth}\raggedright
Espanhol
\end{minipage} \\
\begin{minipage}[b]{\linewidth}\raggedright
\textbf{Manual de la Teneper}
\end{minipage} & \begin{minipage}[b]{\linewidth}\raggedright
\textbf{Waldo Vieira}
\end{minipage} & \begin{minipage}[b]{\linewidth}\raggedright
Espanhol
\end{minipage} \\
\begin{minipage}[b]{\linewidth}\raggedright
\textbf{Glossário Francês-Português}
\end{minipage} & \begin{minipage}[b]{\linewidth}\raggedright
\textbf{Org. Marlise Combet et al}
\end{minipage} & \begin{minipage}[b]{\linewidth}\raggedright
Francês
\end{minipage} \\
\begin{minipage}[b]{\linewidth}\raggedright
\textbf{Courage to Evolve}
\end{minipage} & \begin{minipage}[b]{\linewidth}\raggedright
\textbf{Luciano Vicenzi}
\end{minipage} & \begin{minipage}[b]{\linewidth}\raggedright
Inglês
\end{minipage} \\
\begin{minipage}[b]{\linewidth}\raggedright
\textbf{Megastrongtrait}
\end{minipage} & \begin{minipage}[b]{\linewidth}\raggedright
\textbf{Dayane Rossa}
\end{minipage} & \begin{minipage}[b]{\linewidth}\raggedright
Inglês
\end{minipage} \\
\begin{minipage}[b]{\linewidth}\raggedright
\textbf{Anti Energetic Rubbish}
\end{minipage} & \begin{minipage}[b]{\linewidth}\raggedright
\textbf{Katia Arakaki}
\end{minipage} & \begin{minipage}[b]{\linewidth}\raggedright
Inglês
\end{minipage} \\
\begin{minipage}[b]{\linewidth}\raggedright
\textbf{Life Goes On}
\end{minipage} & \begin{minipage}[b]{\linewidth}\raggedright
\textbf{Betânia Abreu}
\end{minipage} & \begin{minipage}[b]{\linewidth}\raggedright
Inglês
\end{minipage} \\
\begin{minipage}[b]{\linewidth}\raggedright
\textbf{Manuale del Ceneper}
\end{minipage} & \begin{minipage}[b]{\linewidth}\raggedright
\textbf{Waldo Vieira}
\end{minipage} & \begin{minipage}[b]{\linewidth}\raggedright
Italiano
\end{minipage} \\
\begin{minipage}[b]{\linewidth}\raggedright
\textbf{Manuale della Proesis}
\end{minipage} & \begin{minipage}[b]{\linewidth}\raggedright
\textbf{Waldo Vieira}
\end{minipage} & \begin{minipage}[b]{\linewidth}\raggedright
Italiano
\end{minipage} \\
\begin{minipage}[b]{\linewidth}\raggedright
\textbf{Il Piccolo Ricercatore}
\end{minipage} & \begin{minipage}[b]{\linewidth}\raggedright
\textbf{Débora Klippel}
\end{minipage} & \begin{minipage}[b]{\linewidth}\raggedright
Italiano
\end{minipage} \\
\begin{minipage}[b]{\linewidth}\raggedright
\textbf{The English-Japanese Mini-Glossary of Conscientiology Terms}
\end{minipage} & \begin{minipage}[b]{\linewidth}\raggedright
\textbf{Orgs. Keiko Asaoka et al}
\end{minipage} & \begin{minipage}[b]{\linewidth}\raggedright
Japonês
\end{minipage} \\
\begin{minipage}[b]{\linewidth}\raggedright
\textbf{Barbara zboară către stele}
\end{minipage} & \begin{minipage}[b]{\linewidth}\raggedright
\textbf{Jayme Pereira}
\end{minipage} & \begin{minipage}[b]{\linewidth}\raggedright
Romeno
\end{minipage} \\
\begin{minipage}[b]{\linewidth}\raggedright
\textbf{Evoluția noastră}
\end{minipage} & \begin{minipage}[b]{\linewidth}\raggedright
\textbf{Waldo Vieira}
\end{minipage} & \begin{minipage}[b]{\linewidth}\raggedright
Romeno
\end{minipage} \\
\midrule\noalign{}
\endhead
\bottomrule\noalign{}
\endlastfoot
\end{longtable}

\textbf{\hl{REIMPRESSÕES DE OBRAS}}

\begin{longtable}[]{@{}
  >{\raggedright\arraybackslash}p{(\linewidth - 2\tabcolsep) * \real{0.7337}}
  >{\raggedright\arraybackslash}p{(\linewidth - 2\tabcolsep) * \real{0.2663}}@{}}
\toprule\noalign{}
\begin{minipage}[b]{\linewidth}\centering
\textbf{Título}
\end{minipage} & \begin{minipage}[b]{\linewidth}\centering
\textbf{Autor}
\end{minipage} \\
\begin{minipage}[b]{\linewidth}\raggedright
\textbf{A Vida Segue: Diário de Experiências Projetivas}
\end{minipage} & \begin{minipage}[b]{\linewidth}\raggedright
\textbf{Betânia Abreu}
\end{minipage} \\
\begin{minipage}[b]{\linewidth}\raggedright
\textbf{Antibagulhismo Energético}
\end{minipage} & \begin{minipage}[b]{\linewidth}\raggedright
\textbf{Kátia Arakaki}
\end{minipage} \\
\begin{minipage}[b]{\linewidth}\raggedright
\textbf{Autoexperimentação conscienciológica: Método dos Autotestes Experienciais, Pró-evolutivos e Multidimensionais}
\end{minipage} & \begin{minipage}[b]{\linewidth}\raggedright
\textbf{Alexandre Zaslavsky}
\end{minipage} \\
\begin{minipage}[b]{\linewidth}\raggedright
\textbf{Autopesquisa Conscienciológicas: Práticas e Ferramentas}
\end{minipage} & \begin{minipage}[b]{\linewidth}\raggedright
\textbf{Beatriz Tenius e Tatiana Lopes}
\end{minipage} \\
\begin{minipage}[b]{\linewidth}\raggedright
\textbf{Competências Parapsíquicas: Técnicas para o Desenvolvimento do Parapsiquismo Interassistencial}
\end{minipage} & \begin{minipage}[b]{\linewidth}\raggedright
\textbf{Almir Justi, Amin Lascani, Dayane Rossa}
\end{minipage} \\
\begin{minipage}[b]{\linewidth}\raggedright
\textbf{Consciência Centrada na Assistência}
\end{minipage} & \begin{minipage}[b]{\linewidth}\raggedright
\textbf{Flávia Rogick}
\end{minipage} \\
\begin{minipage}[b]{\linewidth}\raggedright
\textbf{Contrapontos do Parapsiquismo: Superação do Assédio Intercosnciencial rumo à Desassedialidade Permanente Total}
\end{minipage} & \begin{minipage}[b]{\linewidth}\raggedright
\textbf{Cirlene Couto}
\end{minipage} \\
\begin{minipage}[b]{\linewidth}\raggedright
\textbf{Descrenciograma: Fundamentos e Teáticas}
\end{minipage} & \begin{minipage}[b]{\linewidth}\raggedright
\textbf{Oswaldo Vernet}
\end{minipage} \\
\begin{minipage}[b]{\linewidth}\raggedright
\textbf{Desenvolvimento Conscienciográfico}
\end{minipage} & \begin{minipage}[b]{\linewidth}\raggedright
\textbf{Tatiana Lopes}
\end{minipage} \\
\begin{minipage}[b]{\linewidth}\raggedright
\textbf{Diário de Experiências Cognopolitanas}
\end{minipage} & \begin{minipage}[b]{\linewidth}\raggedright
\textbf{Ermânia Ribeiro}
\end{minipage} \\
\begin{minipage}[b]{\linewidth}\raggedright
\textbf{Energias: Você Percebe, Utiliza e Doa de modo Eficaz}
\end{minipage} & \begin{minipage}[b]{\linewidth}\raggedright
\textbf{Maria Tereza Bolzan}
\end{minipage} \\
\begin{minipage}[b]{\linewidth}\raggedright
\textbf{Estado Vibracional: Vivências e Autoqualificação}
\end{minipage} & \begin{minipage}[b]{\linewidth}\raggedright
\textbf{Victor Bolfe}
\end{minipage} \\
\begin{minipage}[b]{\linewidth}\raggedright
\textbf{Livros dos Credores Grupocármicos}
\end{minipage} & \begin{minipage}[b]{\linewidth}\raggedright
\textbf{Ernani Brito, Rosemary Sales e Sandra Tornieri}
\end{minipage} \\
\begin{minipage}[b]{\linewidth}\raggedright
\textbf{Manual do Materpensene: a Síntese da Consciência}
\end{minipage} & \begin{minipage}[b]{\linewidth}\raggedright
\textbf{Guilherme Kunz}
\end{minipage} \\
\begin{minipage}[b]{\linewidth}\raggedright
\textbf{Megatrafor: Estudo do Maior Consciencial sob a Ótica da Multiexistencialidade}
\end{minipage} & \begin{minipage}[b]{\linewidth}\raggedright
\textbf{Dayane Rossa}
\end{minipage} \\
\begin{minipage}[b]{\linewidth}\raggedright
\textbf{Nossa Evolução}
\end{minipage} & \begin{minipage}[b]{\linewidth}\raggedright
\textbf{Waldo Vieira}
\end{minipage} \\
\begin{minipage}[b]{\linewidth}\raggedright
\textbf{Onde a Religião Termina}
\end{minipage} & \begin{minipage}[b]{\linewidth}\raggedright
\textbf{Marcelo da Luz}
\end{minipage} \\
\begin{minipage}[b]{\linewidth}\raggedright
\textbf{Prescrição para o Autodesassédio}
\end{minipage} & \begin{minipage}[b]{\linewidth}\raggedright
\textbf{Maximiliano Haymann}
\end{minipage} \\
\begin{minipage}[b]{\linewidth}\raggedright
\textbf{Projeções Assistenciais}
\end{minipage} & \begin{minipage}[b]{\linewidth}\raggedright
\textbf{Marilza de Andrade}
\end{minipage} \\
\begin{minipage}[b]{\linewidth}\raggedright
\textbf{Retrocognições: Pesquisa da Memória de Vivências Passadas}
\end{minipage} & \begin{minipage}[b]{\linewidth}\raggedright
\textbf{Wagner Alegretti}
\end{minipage} \\
\begin{minipage}[b]{\linewidth}\raggedright
\textbf{Sem Medo da Morte: Construindo uma Realidade Multidimensional}
\end{minipage} & \begin{minipage}[b]{\linewidth}\raggedright
\textbf{Vera Hoffmann}
\end{minipage} \\
\begin{minipage}[b]{\linewidth}\raggedright
\textbf{Thesaurus Terminológico da Conscienciologia em Português}
\end{minipage} & \begin{minipage}[b]{\linewidth}\raggedright
\textbf{Eliane Bianchi Wojslaw, Rita Sawaya, Marina Thomaz, Mércia Oliveira e Augusto Freire}
\end{minipage} \\
\begin{minipage}[b]{\linewidth}\raggedright
\textbf{Tridotação Consciencial: Teática Inversiva}
\end{minipage} & \begin{minipage}[b]{\linewidth}\raggedright
\textbf{Lucimara Ribas Frederico}
\end{minipage} \\
\begin{minipage}[b]{\linewidth}\raggedright
\textbf{Viagens Internacionais: o Nomadismo da Conscienciologia}
\end{minipage} & \begin{minipage}[b]{\linewidth}\raggedright
\textbf{Kátia Arakaki}
\end{minipage} \\
\begin{minipage}[b]{\linewidth}\raggedright
\textbf{Vontade: Consciência Inteira}
\end{minipage} & \begin{minipage}[b]{\linewidth}\raggedright
\textbf{Dulce Daou}
\end{minipage} \\
\midrule\noalign{}
\endhead
\bottomrule\noalign{}
\endlastfoot
\end{longtable}

\textbf{\hl{PUBLICAÇÕES EM E-BOOKS E AUDIOBOOKS}}

\begin{longtable}[]{@{}
  >{\raggedright\arraybackslash}p{(\linewidth - 2\tabcolsep) * \real{0.7337}}
  >{\raggedright\arraybackslash}p{(\linewidth - 2\tabcolsep) * \real{0.2663}}@{}}
\toprule\noalign{}
\begin{minipage}[b]{\linewidth}\centering
\textbf{Título}
\end{minipage} & \begin{minipage}[b]{\linewidth}\centering
\textbf{Formato}
\end{minipage} \\
\begin{minipage}[b]{\linewidth}\raggedright
\textbf{História do Parapsiquismo}
\end{minipage} & \begin{minipage}[b]{\linewidth}\raggedright
\emph{e-book}
\end{minipage} \\
\begin{minipage}[b]{\linewidth}\raggedright
\textbf{Acoplamento Energético}
\end{minipage} & \begin{minipage}[b]{\linewidth}\raggedright
\emph{e-book}
\end{minipage} \\
\begin{minipage}[b]{\linewidth}\raggedright
\textbf{Contrapontos do Parapsiquismo}
\end{minipage} & \begin{minipage}[b]{\linewidth}\raggedright
\emph{e-book}
\end{minipage} \\
\begin{minipage}[b]{\linewidth}\raggedright
\textbf{Teoria e Prática da Experiência Fora do Corpo}
\end{minipage} & \begin{minipage}[b]{\linewidth}\raggedright
\emph{e-book}
\end{minipage} \\
\begin{minipage}[b]{\linewidth}\raggedright
\textbf{Glosario de Términos Esenciales de la Concienciología}
\end{minipage} & \begin{minipage}[b]{\linewidth}\raggedright
\emph{e-book}
\end{minipage} \\
\begin{minipage}[b]{\linewidth}\raggedright
\textbf{Conscienciologia: breve Introdução à Ciência da Consciência}
\end{minipage} & \begin{minipage}[b]{\linewidth}\raggedright
\emph{e-book}
\end{minipage} \\
\begin{minipage}[b]{\linewidth}\raggedright
\textbf{A Consciência Centrada na Assistência}
\end{minipage} & \begin{minipage}[b]{\linewidth}\raggedright
\emph{e-book}
\end{minipage} \\
\begin{minipage}[b]{\linewidth}\raggedright
\textbf{Manual da Proéxis}
\end{minipage} & \begin{minipage}[b]{\linewidth}\raggedright
\emph{audiobook}
\end{minipage} \\
\begin{minipage}[b]{\linewidth}\raggedright
\textbf{O que é Conscienciologia?}
\end{minipage} & \begin{minipage}[b]{\linewidth}\raggedright
\emph{audiobook}
\end{minipage} \\
\midrule\noalign{}
\endhead
\bottomrule\noalign{}
\endlastfoot
\end{longtable}

\textbf{PUBLICAÇÕES DE PERIÓDICOS CONSCIENCIOLÓGICOS}

\begin{longtable}[]{@{}
  >{\raggedright\arraybackslash}p{(\linewidth - 2\tabcolsep) * \real{0.7337}}
  >{\raggedright\arraybackslash}p{(\linewidth - 2\tabcolsep) * \real{0.2663}}@{}}
\toprule\noalign{}
\begin{minipage}[b]{\linewidth}\centering
\textbf{Título}
\end{minipage} & \begin{minipage}[b]{\linewidth}\centering
\textbf{Quantidade}
\end{minipage} \\
\begin{minipage}[b]{\linewidth}\raggedright
\textbf{Conscientia}
\end{minipage} & \begin{minipage}[b]{\linewidth}\raggedright
\textbf{5 edições}
\end{minipage} \\
\begin{minipage}[b]{\linewidth}\raggedright
\textbf{Scriptor}
\end{minipage} & \begin{minipage}[b]{\linewidth}\raggedright
\textbf{1 edição}
\end{minipage} \\
\begin{minipage}[b]{\linewidth}\raggedright
\textbf{Interparadigmas}
\end{minipage} & \begin{minipage}[b]{\linewidth}\raggedright
\textbf{1 edição}
\end{minipage} \\
\begin{minipage}[b]{\linewidth}\raggedright
\textbf{Editares}
\end{minipage} & \begin{minipage}[b]{\linewidth}\raggedright
\textbf{1 edição}
\end{minipage} \\
\begin{minipage}[b]{\linewidth}\raggedright
\textbf{Holothecology (Suplemento)}
\end{minipage} & \begin{minipage}[b]{\linewidth}\raggedright
\textbf{1 edição}
\end{minipage} \\
\begin{minipage}[b]{\linewidth}\raggedright
\textbf{Gescons}
\end{minipage} & \begin{minipage}[b]{\linewidth}\raggedright
\textbf{1 edição}
\end{minipage} \\
\begin{minipage}[b]{\linewidth}\raggedright
\textbf{Holotecologia}
\end{minipage} & \begin{minipage}[b]{\linewidth}\raggedright
\textbf{1 edição}
\end{minipage} \\
\midrule\noalign{}
\endhead
\bottomrule\noalign{}
\endlastfoot
\end{longtable}

RESUMO DO BIÊNIO

\textbf{Voluntários: Quem faz a Editares Acontecer}

\emph{Por Ana Claudia Prado e Magda Stapf Amancio}

A Editares é resultado do \textbf{esforço coletivo de consciências engajadas na tarefa do esclarecimento (tares)}. Cada livro, cada publicação e cada projeto só são possíveis graças à dedicação dos voluntários que integram nossas equipes. A seguir, apresentamos a listagem completa dos atuais voluntários, organizados por áreas de atuação, como forma de reconhecer e valorizar o trabalho diário que sustenta o trabalho editorial da instituição.

\emph{``Voluntariado: autodemonstração interassistencial.''} (Vieira, 2014, p. 1.715)

\begin{longtable}[]{@{}
  >{\raggedright\arraybackslash}p{(\linewidth - 2\tabcolsep) * \real{0.5000}}
  >{\raggedright\arraybackslash}p{(\linewidth - 2\tabcolsep) * \real{0.5000}}@{}}
\toprule\noalign{}
\begin{minipage}[b]{\linewidth}\centering
\textbf{Equipes}
\end{minipage} & \begin{minipage}[b]{\linewidth}\centering
\textbf{Voluntários}
\end{minipage} \\
\begin{minipage}[b]{\linewidth}\raggedright
Acessibilidade
\end{minipage} & \begin{minipage}[b]{\linewidth}\raggedright
Fabiane Cattai; Lurdes Sousa; Rui Sousa; Sónia Luginger.
\end{minipage} \\
\begin{minipage}[b]{\linewidth}\raggedright
Biblioteconomia
\end{minipage} & \begin{minipage}[b]{\linewidth}\raggedright
Beatriz Cestari; Betânia Abreu.
\end{minipage} \\
\begin{minipage}[b]{\linewidth}\raggedright
Comercial
\end{minipage} & \begin{minipage}[b]{\linewidth}\raggedright
Renata Pialarissi; Ricardo Rezende.
\end{minipage} \\
\begin{minipage}[b]{\linewidth}\raggedright
Conselheiros Editoriais
\end{minipage} & \begin{minipage}[b]{\linewidth}\raggedright
Ana Cláudia Prado; Ana Mazzonetto; Cecília Roma; Cristina Bornia; Cristina Ellwanger; José Ricardo Gomes; Leonardo Rodrigues; Liege Trentin; Magda Stapf; Meracilde Daroit; Patrícia Pialarissi; Roberta Bouchardet.
\end{minipage} \\
\begin{minipage}[b]{\linewidth}\raggedright
Coordenação
\end{minipage} & \begin{minipage}[b]{\linewidth}\raggedright
Ana Claudia Prado; Magda Stapf.
\end{minipage} \\
\begin{minipage}[b]{\linewidth}\raggedright
Diagramação de ebooks
\end{minipage} & \begin{minipage}[b]{\linewidth}\raggedright
Erik Scardino; Luiz Eduardo Menezes; Márcia Perrusi; Maria Koltum; Rosana Cardoso.
\end{minipage} \\
\begin{minipage}[b]{\linewidth}\raggedright
Diagramação de livros
\end{minipage} & \begin{minipage}[b]{\linewidth}\raggedright
Daniel Ronque; Eduardo Santana; João Feliciano; Kátia Ávila; Liliana Ferreira.
\end{minipage} \\
\begin{minipage}[b]{\linewidth}\raggedright
Diagramação Gráfica de periódicos
\end{minipage} & \begin{minipage}[b]{\linewidth}\raggedright
Ana Claudia Prado; Luciano Melo; Tatiane Mendonça.
\end{minipage} \\
\begin{minipage}[b]{\linewidth}\raggedright
Editoração Internacional
\end{minipage} & \begin{minipage}[b]{\linewidth}\raggedright
Cecília Roma; Marcos Paiva; Roberta Bouchardet.
\end{minipage} \\
\begin{minipage}[b]{\linewidth}\raggedright
Financeiro/Administrativo
\end{minipage} & \begin{minipage}[b]{\linewidth}\raggedright
Alex Sarmento; Amanda Vieira;

Conselho Fiscal (Carlos Moreno, Izabel Conceição e Maria Zilá).
\end{minipage} \\
\begin{minipage}[b]{\linewidth}\raggedright
Lives (transmissões)
\end{minipage} & \begin{minipage}[b]{\linewidth}\raggedright
Blandina Monteiro; Lurdes Sousa; Rui Sousa.
\end{minipage} \\
\begin{minipage}[b]{\linewidth}\raggedright
Mídias Sociais
\end{minipage} & \begin{minipage}[b]{\linewidth}\raggedright
Paula Gabriella Barbosa.
\end{minipage} \\
\begin{minipage}[b]{\linewidth}\raggedright
Pareceristas
\end{minipage} & \begin{minipage}[b]{\linewidth}\raggedright
Amanda Vieira; Dayane Rossa; João Paulo; Liliana Ferreira; Márcia Perrusi; Rúbia Henning; Conselheiros Editoriais.
\end{minipage} \\
\begin{minipage}[b]{\linewidth}\raggedright
Revisores de Diagramação e Prova Final
\end{minipage} & \begin{minipage}[b]{\linewidth}\raggedright
Niciano Vilas Boas.
\end{minipage} \\
\begin{minipage}[b]{\linewidth}\raggedright
Revisores Linguístico-Textual
\end{minipage} & \begin{minipage}[b]{\linewidth}\raggedright
Bruno Camargo; Liliane Sakakima; Tatiane Mendonça.
\end{minipage} \\
\begin{minipage}[b]{\linewidth}\raggedright
Voluntariado
\end{minipage} & \begin{minipage}[b]{\linewidth}\raggedright
Blandina Monteiro; Magda Stapf.
\end{minipage} \\
\begin{minipage}[b]{\linewidth}\raggedright
Website
\end{minipage} & \begin{minipage}[b]{\linewidth}\raggedright
Leonardo Ribeiro.
\end{minipage} \\
\midrule\noalign{}
\endhead
\bottomrule\noalign{}
\endlastfoot
\end{longtable}

ATUALIZAÇÕES

\textbf{Nova Escola de Editores impulsiona formação editoriológica na Editares}

Em julho de 2025, teve início a \emph{Escola de Editores Conscienciológicos.} Trata-se de iniciativa inédita, objetivando formar e qualificar voluntários do \emph{Conselho Editorial}. O curso foi formulado pelos atuais conselheiros, ex-gestores da Editares, voluntários da UNICIN e da CCCI com experiência no processo editorial. \textbf{A meta é ampliar o número de editores, qualificar as publicações, otimizar o fluxo editorial e ampliar a interação entre os editores mais experientes com os novos}.

A \emph{Escola} responde à crescente demanda de autores e livros. A proposta ganhou corpo com apoio da UNICIN, dentro do movimento de reestruturação institucional realizado em 2025. Com 11 módulos quinzenais aos domingos, em formato híbrido, o curso vai até dezembro deste ano, combinando teoria e prática editorial.

A expectativa é consolidar a formação como etapa contínua da atuação editorial na CCCI, atraindo novos voluntários. A turma piloto servirá como base para ajustes e institucionalização futura da Escola, que poderá ter novas edições ou formato gravado.

Estamos empolgados com a expectativa do que pode resultar em termos de mudança de patamar do conselho editorial!

\emph{Por Roberta Bouchardet \& Amanda Vieira}

\emph{Coordenadora \& Monitora da Escola de Editores}

\emph{{[}foto das autoras{]}}

ATUALIZAÇÕES

\textbf{Projetos digitais modernizam a comunicação editorial da Editares}

Nos últimos anos, a comunicação da Editares passou por importantes atualizações. O trabalho de divulgação não se restringe apenas ao anúncio de novos lançamentos, mas se dedica principalmente a valorizar e compartilhar os conteúdos das gescons publicadas. A proposta é \textbf{oferecer ao futuro leitor amostra prática} do que encontrará nas obras, ampliando o alcance de cada livro.

Nesse sentido, a Editares tem investido em diferentes formatos para tornar o conteúdo mais acessível e interativo. Entre eles, destacam-se os vídeos gravados pelos autores, que aproximam o leitor da vivência pessoal de quem escreveu a obra, e os carrosséis produzidos a partir das resenhas, que apresentam ideias centrais de forma visual e objetiva. Assim, cada publicação nas redes sociais se torna uma oportunidade de aprofundar a compreensão e estimular o interesse pela leitura.

Além disso, a Editares tem promovido \textbf{ações que fortalecem o vínculo entre os livros e a comunidade conscienciológica.} Entre elas estão:

\begin{enumerate}
\def\labelenumi{\arabic{enumi}.}
\item
  Dicas de leitura associadas a cursos em andamento, criando conexões pertinentes ao aluno e favorecendo a interação com as instituições parceiras;
\item
  Campanhas promocionais em datas especiais, como a promoção do livro Comunicação Evolutiva durante o aniversário de 20 anos da Comunicons.
\end{enumerate}

Essas iniciativas têm tornado a Editares mais presente e levam o conteúdo das gescons a um público cada vez maior. Ao divulgar trechos, ideias e reflexões dos livros, a Editares reforça seu papel como ponte entre o conhecimento conscienciológico e o leitor, expandindo a interassistência por meio da leitura.

\emph{Por Paula Gabriella}

\emph{Voluntária da Comunicação da Editares}

\emph{{[}Foto da autora{]}}

{[}INSERIR FOTO QUE ESTÁ NA PASTA MATÉRIA 5{]}

ATUALIZAÇÕES

\textbf{Editares lança revista científica especializada em Editoriologia e Publicaciologia}

A Associação Internacional Editares dedicada à editoração e publicação de gestações conscienciais, anunciou em novembro de 2025 o lançamento da \textbf{Revista Editares,} publicação científica voltada exclusivamente às especialidades de \emph{Editoriologia e Publicaciologia.} O lançamento ocorreu durante evento temático sobre essas áreas de estudo: \textbf{II Simpósio Editoriologia e Publicaciologia.}

A novidade representa um avanço em relação à edição especial da \emph{Revista Gescons} nº 5, publicada em 2023, quando voluntários da Editares apresentaram, por meio de artigos científicos, os bastidores da produção editorial conscienciológica.

A escolha de um novo nome não se restringe a uma mudança formal. A decisão editorial visa demarcar, de modo preciso, o \textbf{caráter científico da publicação} e a construção de novos conhecimentos a partir de artigos técnicos nas duas especialidades.

Com a criação da \emph{Revista Editares,} a instituição passa a separar claramente os campos de atuação: enquanto a nova publicação concentra análises científicas, a \emph{Revista Gescons} continua sendo o espaço para registro de lançamentos de livros e outras pontuações editoriais e administrativas.

A iniciativa reforça o compromisso da Editares com o profissionalismo editorial, legitima processos de revisão mais exigentes de revisão por pares, fortalece a especialização temática e consolida a identidade científica da instituição.

O primeiro número da \emph{Revista Editares} reúne \textbf{treze artigos inéditos,} abordando desde os meandros do fluxo editorial até as vivências de voluntários no processo de editoração e publicação de obras conscienciológicas, sempre à luz do Paradigma Consciencial.

\emph{``Este é um momento de atualização do processo editorial da Editares, no qual reafirmamos publicamente nosso compromisso com a produção científica nas especialidades de Editoriologia e Publicaciologia''}, ressaltam as editoras Ana Claudia Prado e Ana Mazzonetto.

\emph{Por Ana Claudia Prado \& Ana Mazzonetto}

\emph{Editoras da Revista Editares}

\emph{{[}Foto das autoras{]}}

\emph{{[}INSERIR MOCKUP DA REVISTA QUE ESTÁ NA PASTA MATÉRIA 6{]}}

ATUALIZAÇÕES

\textbf{Editares e IIPC retomam parceria para ampliar a venda de livros}

Neste ano, a Editares, coordenada pelas voluntárias Ana Claudia Prado e Magda Stapf, celebra acordo exitoso com a coordenação geral e equipe do IIPC, representados pelos Coordenadores Gerais Gabriel Araújo e Ailton Maia, pelo qual \textbf{as obras conscienciológicas} estão sendo disponibilizadas para compra nos acervos dos \emph{Centros Educacionais de Autopesquisa} (CEAs) do \emph{Instituto Internacional de Projeciologia e Cons­cienciologia} (IIPC).

A renovação dessa parceria além de possibilitar a melhoria logística da distri­buição comercial desses livros, auxiliará na ampliação da difusão tarística dos livros pu­blicados de autores conscienciológicos ao permitir o acesso físico às gestações conscien­ciais para os visitantes e voluntários nos CEAs do IIPC.

\begin{quote}
\emph{Por: Ricardo Rezende}

\emph{Voluntário do Setor Comercial da Editares}
\end{quote}

ATUALIZAÇÕES

\textbf{Grupos de trabalho com a UNICIN traçam planos para fortalecer a atuação da Editares}

\emph{Por Amanda Vieira}

\emph{Voluntária da Editares}

Em 2025, a Editares deu importante passo estratégico para potencializar sua atuação na produção e distribuição de gescons. Em parceria com a UNICIN e a UNIESCON, foram criados \textbf{dois grupos de trabalho (GTs)} voltados à reformulação de processos e à implantação de novas iniciativas, com foco em dois pontos principais: o fluxo editorial e a venda de livros.

Os encontros reuniram voluntários de diferentes ICs e áreas de atuação, evidenciando a importância da intercooperação na Comunidade Conscienciológica. Dessa troca nasceram dois GTs específicos: GT Editorial e GT Comercial.

\textbf{DIAGNÓSTICO, PLANEJAMENTO E AÇÃO}

Durante o primeiro semestre, os subgrupos trabalharam em diagnósticos detalhados, mapeando gargalos, levantando necessidades e propondo soluções. A partir desse estudo, foram elaborados planos de ação que começaram a ser implementados no segundo semestre de 2025.

No \textbf{GT Comercial,} as iniciativas se voltaram à expansão da presença da Editares, com novos processos de marketing, estratégias de divulgação e ações de lançamento de livros. Já no \textbf{GT Editorial,} o destaque foi a criação da \emph{Escola de Editores,} um projeto voltado à capacitação de voluntários para ampliar a qualidade técnica e a celeridade na editoração de obras conscienciológicas.

\textbf{A FORÇA DO VOLUNTARIADO E A UNIÃO DAS ICS}

O movimento reforça a relevância da união entre as Instituições Conscienciocêntricas como fator-chave para a expansão da Conscienciologia. A intercooperação entre Editares, UNICIN e UNIESCON potencializa a produção e a difusão das gescons, beneficiando pesquisadores, autores e leitores.

Nesse contexto, o voluntariado se destaca como força motriz da \emph{reurbanização extrafísica} (reurbex), atuando na materialização de projetos que impactam diretamente a evolução grupal. O livro, enquanto ferramenta de autopesquisa e recomposição grupocármica, cumpre papel central no esclarecimento e na ampliação da lucidez planetária.

Com essas iniciativas, a Editares reafirma seu compromisso com a qualidade editorial, a celeridade dos processos e a ampliação da distribuição das obras conscienciológicas. Mais do que uma reorganização interna, trata-se de um movimento coletivo que coloca o livro e o conhecimento técnico ao serviço da evolução pessoal e grupal.

ENTREVISTAS

\begin{enumerate}
\def\labelenumi{\arabic{enumi}.}
\item
  \textbf{\emph{Mutatis Mutandis:} Teoria e Prática da Reciclagem Existencial}
\item
  \textbf{Megapensenes Trivocabulares Pró-evolutivos}
\item
  \textbf{Epicentrismo Consciencial: Casuísticas Recinológicas}
\item
  \textbf{Interassistência: Teoria e Prática sob a Ótica da Conscienciologia}
\item
  \textbf{Gratidão: Reconhecer, Expressar e Retribuir}
\item
  \textbf{Autoexperimentação Conscienciológica}
\item
  \textbf{O Calidoscópio da evolução consciencial}
\item
  \textbf{Ortoprincípios Norteadores da Invéxis}
\item
  \textbf{Ações Antifanatismo}
\item
  \textbf{Autoidentificação Proexológica}
\item
  \textbf{The English-Japanese Mini-Glossary of Conscientiology Terms}
\item
  \textbf{Eitologia: Teática Inversiva}
\item
  \textbf{Retomada de Tarefa: Reencontro com o Paradever Intermissivo}
\item
  \textbf{Autoinversão Existencial: Compreensão e Vivência}
\item
  \textbf{Estado Vibracional: 100 Perguntas e Respostas}
\item
  \textbf{Migrações Proexológicas Internacionais}
\item
  \textbf{Autoconscientização Grafopensênica}
\item
  \textbf{Tanatofobia: da Ressignificação da Existência à Superação do\,Medo\,da\,Morte}
\item
  \textbf{Diário de um Iniciante na Tenepes}
\item
  \textbf{Imobilidade Física Vígil: Autopesquisa Experimentológica}
\item
  \textbf{Seleção Consciencial}
\item
  \textbf{Libertações Evolutivas: Trajetória Semperaprendente Sob Olhar Conscienciológico}
\item
  \textbf{Da Labilidade Parapsíquica A Assistencialidade Lúcida: Uma Trajetória Evolutiva}
\item
  \textbf{A Lógica do Pensene}
\item
  \textbf{Coadjuvantes da Invéxis}
\item
  \textbf{Dicionário de Paradireito: Exemplarium}
\end{enumerate}

\textbf{BALANÇO DA EDITARES 21 ANOS}

\textbf{2004}

3 lançamentos

\textbf{2005}

1 lançamento

\textbf{2006}

3 lançamentos

1 reedição

\textbf{2007}

3 lançamentos

1 reedição

\textbf{2008}

1 lançamento

3 reedições

\textbf{2009}

2 lançamentos

\textbf{2010}

3 lançamentos

3 reedições

\textbf{2011}

5 lançamentos

4 reedições

\textbf{2012}

3 lançamentos

1 reimpressão

1 reedições

\textbf{2013}

10 lançamentos

2 reimpressões

7 reedições

\textbf{2014}

10 lançamentos

1 reimpressão

1 reedição

\textbf{2015}

14 lançamentos

2 reimpressões

7 reedições

7 e-books

\textbf{2016}

10 lançamentos

1 reedição

10 e-books

\textbf{2017}

13 lançamentos

5 reedições

9 e-books

\textbf{2018}

14 lançamentos

11 reimpressões

1 e-book

\textbf{2019}

16 lançamentos

15 reimpressões

4 e-books

\textbf{2020}

21 lançamentos

5 reimpressões

8 e-books

\textbf{2021}

29 lançamentos

28 reimpressões

1 e-book

\textbf{2022}

23 lançamentos

10 reimpressões

1 reedição

\textbf{2023}

35 lançamentos

16 reimpressões

3 e-books

\textbf{\hl{2024}}

\textbf{\hl{2025}}

\begin{quote}
\textbf{DIVULGAÇÃO UNIESCON (FRENTE)}

\textbf{DIVULGAÇÃO AMIGOS DA ENCICLOPÉDIA (VERSO)}

\textbf{NÃO ACREDITE EM NADA,}

\textbf{NEM MESMO NO QUE LER NESTA PUBLICAÇÃO.}

\textbf{EXPERIMENTE.}

\textbf{TENHA AS SUAS EXPERIÊNCIAS PESSOAIS.}

\textbf{DO NOT BELIEVE IN ANYTHING,}

\textbf{NOT EVEN IN WHAT YOU READ IN THIS PUBLICATION. EXPERIMENT.}

\textbf{HAVE YOUR OWN EXPERIENCES.}

\textbf{NO CREA EN NADA,}

\textbf{NI SIQUIERA EN LAS INFORMACIONES EXPUESTAS EN ESTE CATALOG.}

\textbf{EXPERIMENTE.}

\textbf{TENGA SUS PRÓPRIAS EXPERIENCIAS.}

\textbf{CONTRACAPA (SOMENTE O LOGO DA EDITARES)}
\end{quote}
